% install texlive-science to enable algorithm package %

\documentclass{article} % Use the report class instead of article
\usepackage{titlesec}
\usepackage{graphicx}
\usepackage{hyperref}
\usepackage{listings}
\usepackage{color}
\usepackage{amsmath} % Add this line to use \text
\usepackage{tabularx}
\usepackage{algorithm}
\usepackage{tcolorbox}
\usepackage[noend]{algpseudocode}

\definecolor{dkgreen}{rgb}{0,0.6,0}
\definecolor{gray}{rgb}{0.5,0.5,0.5}
\definecolor{mauve}{rgb}{0.58,0,0.82}

\lstset{frame=tb,
  language=Java,
  aboveskip=3mm,
  belowskip=3mm,
  showstringspaces=false,
  columns=flexible,
  basicstyle={\small\ttfamily},
  numbers=none,
  numberstyle=\tiny\color{gray},
  keywordstyle=\color{blue},
  commentstyle=\color{dkgreen},
  stringstyle=\color{mauve},
  breaklines=true,
  breakatwhitespace=true,
  tabsize=3
}

\graphicspath{{./assets/images/}}

\title{%
    \includegraphics[width=0.3\linewidth]{./assets/logo.pdf}\\[20pt]
    \Huge \bfseries Criptografia aplicada \\[10pt]
    \Large DRSA
}
\author{Tiago Silvestre, 103554 \\ Diogo Matos, 102848}
\date{\today}

\begin{document}

\maketitle

\newpage

\tableofcontents

\clearpage

\section{Introduction}
...

\section{Implementation}
\subsection{Random byte generator}
...

\subsection{RSA key-pair generator}

This process is followed in order to generate a RSA key-pair:
\begin{enumerate}
  \item Inputs: exponent, $e$, and the key size.
  \item Generate two random prime numbers, $p$ and $q$, with a size equal to half of the key length.
  In order to generete a random prime number, we use our random byte generator to generate
  a value with the needed size, then we perform a primality test, the value 2 is added to the value until it passes primality test;
  OpenSSL's Big Num \textit{BN\_is\_prime\_fasttest\_ex} is used to perform the primality test. It starts by making a trial division by a set of
  small prime numbers and then at least 128 rounds of Miller-Rabin. This is considered a probability
  test that performs multiple checks to see if the number is composite, if all return negative, the number is considered prime. The process of 
  prime generation has a $2^{-256}$ false positive rate.
  \item Compute $n$, $n=p*q$;
  \item Compute Carmichael's totient function, $y(n)$, for that we compute $$y(n)=\frac{(p-1)(q-1)}{\gcd((p-1),(q-1))}$$.
  After the computation, we check the condition $$gcd(e, y(n)) = 1$$, if false we do not proced;
  \item Compute $d$, the modular inverse, $$d=e^{-1} \bmod{y(n)}$$;
  \item Compute Chinese remainder theorem related constants, $dmp1$, $dmq1$ and $iqmp$. 
  $$dmp1=d \bmod{(p-1)}$$, $$dmq1=d \bmod{(q-1)}$$, $$iqmp1=q \bmod{p}$$;
  \item $n$, $e$, $d$, $p$, $q$, $dmp1$, $dmq1$ and $iqmp1$ are the private key. $n$, $e$ and $d$ are the public key.
\end{enumerate}

All the computed values are used to initialize a OpenSSL RSA key object that is then exported as two PEM files, one 
with the public information and other with the private.

\section{Discussion}
...

\section{Conclusions}

...

\end{document}
